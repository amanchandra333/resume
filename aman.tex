
% Build using XeLaTex

\documentclass[letterpaper]{aman}
\begin{document}

\namesection{Aman}{Chandra}{ \urlstyle{same}\url{http://amanchandra.in} \\
\href{mailto:amanchandra@iitkgp.ac.in}{amanchandra@iitkgp.ac.in} | +918967126457 \\
D-406 Nehru Hall of Residence, IIT Kharagpur, W.B., INDIA -721302
}

%     COLUMN ONE

\begin{minipage}[t]{0.39\textwidth} 

%     EDUCATION

\section{Education} 

\subsection{Indian Institute Of Technology, Kharagpur}
\descript{B.Tech + M.Tech in Biotechnology\\Sophomore}\\
\location{Expected April 2020 | Kharagpur, India \\ CGPA: 7.33/10.0}
\sectionsep


\subsection{St. Michael's High School}
\descript{All India Senior School Certificate}\\
\location{Grad. April 2015|  Patna, India \\Percentage: 95.6\%}
\sectionsep

\subsection{St. Xavier's Hr. Sec. School}
\descript{Certificate of Merit}\\
\location{Grad. April 2013|  Bettiah, India\\CGPA: 10.0/10.0 }
\sectionsep

%     LINKS

\section{Links} 
Github:// \href{https://github.com/amanchandra333}{\custombold{amanchandra333}} \\
LinkedIn://  \href{https://www.linkedin.com/in/amanchandra333}{\custombold{amanchandra333}} \\
Robotix:// \href{https://www.robotix.in/team}{\custombold{team}} \\
\sectionsep

%     COURSEWORK

\section{Coursework}
Programming \& Data Structures\\
Partial Differential Equations \\
The Arduino Platform and C Prog. (Coursera)\\
Control of Mobile Robots (Coursera)\\
Robotics: Aerial Robotics (Coursera)\\
C++ for C Programmers (Coursera) \\
Intro to Java Programming (Udacity)\\
Developing Android Apps (Udacity)\\
\sectionsep


%     SKILLS

\section{Technical Expertise}
\descript{Hardware}\\
ATmega \textbullet{} Arduino \textbullet{} Raspberry Pi\\
%\sectionsep
\descript{Software}\\
Atmel Studio \textbullet Proteus \textbullet SolidWorks \textbullet Android Studio \textbullet After Effects \\
%\sectionsep
\descript{Languages}\\
 C  \textbullet{} C++ \textbullet{} Python \textbullet{} MATLAB  \textbullet{} \LaTeX \textbullet{}  Java \textbullet{} HTML \textbullet Bash\\
%\sectionsep
\descript{Systems}\\
ROS \textbullet{} GIT \textbullet OpenCV \textbullet Gazebo \textbullet Internet of Things\\
\sectionsep



\section{Extra-Curriculars}  
Eloquent Speaker \textbullet{} Avid Quizzer \textbullet{}\\
Table Tennis 
\sectionsep

%     COLUMN TWO

\end{minipage} 
\hfill
\begin{minipage}[t]{0.599\textwidth} 


%     EXPERIENCE

\section{Research Experience/ Projects}

\runsubsection{\href{https://quadrotor-iitkgp.github.io/}{Aerial Robotics Kharagpur}}
\descript{| Control Systems and Embedded Electronics Team}\location{| February 2016 – Present}
\location{\custombold{Guide:}  \href{http://www.facweb.iitkgp.ernet.in/~smsh/}{Prof. Somesh Kumar}}
\vspace{\topsep}
\begin{tightemize}
\item Made Simulink model of quadrotor along with PID control on MATLAB.
\item Designing control systems for a multicopter to take part in IARC 2017.
\item Fabricating safety kill-switch according to IARC guidelines.
\end{tightemize}
\sectionsep

\runsubsection{\href{http://swarm-iitkgp.github.io/}{Swarm Robotics}}
\descript{| Embedded Electronics Team}\location{| October 2016 – Present}
\location{\custombold{Guide:}  \href{http://www.facweb.iitkgp.ernet.in/~smsh/}{Prof. Somesh Kumar}}
\begin{tightemize}
\item Interfaced nRF transceivers with Arduino for communication between multiple robots.
\item Interfaced Xbee modules with Raspberry Pi for robot localization.
\end{tightemize}
\sectionsep

\runsubsection{Motion Imitating and Path Replicating Robot}
\descript{| Project Head \& Instructor}\location{| December 2016}
\begin{tightemize}
\item Accomplished IEEE certified project of making an autonomous robot which could imitate motion of another robot on ATmega16 and Arduino.
\item Mentored a team of 30 students.
\end{tightemize}
\sectionsep

\runsubsection{Step Counter Heading Following Robot}
\descript{| Project Member}\location{| December 2015}
\begin{tightemize}
\item Accomplished IEEE certified project of making a semi- autonomous robot which could follow a human on ATmega32.
\end{tightemize}
\sectionsep

\runsubsection{Hex-Decoding and Path Planning Robot}
\descript{|Team Leader}\location{| January 2016}
\begin{tightemize}
\item Built an autonomous robot which could decode hex-encoded data from IR LEDs and reach its destination, Kshitij ’16, IIT Kharagpur.
\end{tightemize}
\sectionsep

\section{Positions of responsibility}

\runsubsection{\href{https://www.robotix.in/}{Technology Robotix Society}}
\descript{| Sub-head}
\location{| February 2016 - Present}
\begin{tightemize}
\item As part of the university’s official robotics and hobby maker group, conducted the largest robotics related fest in India, Robotix 2016.
\item Spearheaded multiple workshops across India to spread the culture of robotics.
\end{tightemize}
\sectionsep 

\runsubsection{KRAIG}
\descript{| Instructor}
\location{| February 2016 - Present\\
Kharagpur Robotics and Artifical Intellgence Group}
\begin{tightemize}
\item Organized weekly lectures on Manual and Autonomous Robotics for over 200 first year students round the year.
\item Conceptualized 10 IEEE certified projects for over 800 students.
\end{tightemize}
\sectionsep 

\section{Scholastic Achievements}
\begin{tabular}{rl}
Sept, 2016 & $4^{th}$ in Event Sensorous, National Students' Space Challenge'16\\
Jun, 2015 & Qualified JEE Main and Advanced for science \& engineering\\
         &  education entrance in India with percentiles of 98.97 (All India \\
& Rank 3030) and 96.5 (All India Rank 5263) respectively.\\
\end{tabular}


\end{minipage} 
\end{document}  \documentclass[]{article}